%% LyX 1.6.7 created this file.  For more info, see http://www.lyx.org/.
%% Do not edit unless you really know what you are doing.
\documentclass[a4paper,english]{scrbook}
\usepackage[T1]{fontenc}
\usepackage[latin9]{inputenc}
\usepackage{babel}

\usepackage{amsthm}
\usepackage{amsmath}
\usepackage[unicode=true, 
 bookmarks=true,bookmarksnumbered=true,bookmarksopen=true,bookmarksopenlevel=3,
 breaklinks=false,pdfborder={0 0 0},backref=page,colorlinks=false]
 {hyperref}
\hypersetup{pdftitle={The dylan.NET Manual},
 pdfauthor={Dylan Borg},
 pdfsubject={dylan.NET}}

\makeatletter

%%%%%%%%%%%%%%%%%%%%%%%%%%%%%% LyX specific LaTeX commands.
\special{papersize=\the\paperwidth,\the\paperheight}


%%%%%%%%%%%%%%%%%%%%%%%%%%%%%% Textclass specific LaTeX commands.
\numberwithin{equation}{section}
\numberwithin{figure}{section}

\makeatother

\begin{document}

\title{The dylan.NET Manual v.11.1.2}


\author{by Dylan Borg}


\dedication{To those who taught me,to my mum, sister and father.}

\maketitle
\tableofcontents{}


\chapter{The Compiler}


\section{Introduction\label{sec:Introduction}}

This chapter of the manual will speak about the inner workings of
the new dylan.NET compiler. For the language syntax look further down
this manual. The need for a chapter like this has risen becuase of
the new API nature of dylan.NET i.e. now dylan.NET is split in class
libraries each doing a step that transforms a basic form of representation
into a more complex form. The compiler's work is just that, converting
source code written with a basic text editor into an AST, then into
MSIL/CIL that is compatible with .NET 3.5 SP1 or higher and Novell
Mono 2.6.7 or higher.\\
\\
Since Mono is the least common denominator between the two frameworks,
its libraries shall be used in building the compiler. Mono is also
cross-paltform (i.e. works on Windows, Linux and Macintosh OS/X),
hence the need to be compatible with it. Go to the Mono website %
\footnote{http://www.mono-project.com contains info about Mono as well as downloads
for Windows, Mac and Linux.%
} for more info.


\section{The dylan.NET API\label{sec:The-dylan.NET-API}}

The API is split into 4 libraries. These are (a {*} means that the
library is incomplete or not available yet):
\begin{description}
\item [{tokenizer.AST.dll}] Contains all the AST components such as Tokens,
Expression, Statements etc. defined in the dylan.NET language. The
other libraries make heavy use of this library.
\item [{tokenizer.Lexer.dll}] Contains the Lexer components that can turn
a dylan.NET file into statements and tokens.
\item [{tokenizer.Parser.dll{*}}] Contains the Parser components responsible
for the optimization of statements. It can recognize the type of statements
and tokens.
\item [{tokenizer.CodeGen.dll{*}}] Contains the components that turn the
AST into MSIL/CIL code. (Still to be written.) 
\end{description}
The version number for all assemblies should match for a given dylan.NET
distro. The program \textbf{dnc.exe} wraps the 4 libraries and is
the main compiler executable. It also is an example for the use of
the libraries. The libraries and their sources are available from
Gitorious. %
\footnote{http://gitorious.org/dylandotnet/dylandotnet has the latest sources
inside the git repo.%
}


\subsection{AST\label{sub:AST}}

The AST or as I call it, the festival of inheritance contains all
dylan.NET language components. The root namespace is dylan.NET.Tokenizer.AST.
All classes derive from one of the following classes:
\begin{description}
\item [{Token}] A standard dylan.NET token comprising an identifier, literal,
operator etc. All tokens inherit from this class.
\item [{Expr}] A standard dylan.NET expression from which all expressions
are derived.
\item [{Stmt}] A standard statement from which all other statements are
derived. These can be collected in an \textbf{StmtSet}.
\end{description}

\subsection{Lexer\label{sub:Lexer}}

The Lexer is what takes all text source files, splits them into lines
drom which it makes statements and then splits each line into tokens
which it puts inside the corresponding statement. It then store the
set of statements into a statement set for handing over to the Parser.
The dylan.NET lexer is generally string and character aware i.e. it
will not split the token stream when inside a character (e\@.g. 'c')
or when inside a string (e.g. {}``This is a string''). The spaces
in the string used before will not be used to split the string into
tokens as the lexer knows that it is a string literal. The lexer also
has an ingenious system for recognizing operators that are multi-character
such as ++,>=,!=,--, etc. The root namespace is dylan.NET.Tokenizer.Lexer.


\subsection{Parser\label{sub:Parser}}

The Parser os what takes the \textbf{StmtSet} made by the Lexer and
transforms it into the specific statements containing specific tokens.
etc. The decision is done based on the textual value of the tokens
inside the statements. For example a token whose \textbf{Value} field
says {}``object'' gets converted into an \textbf{ObjectTok} which
inherits the class \textbf{TypeTok }~which in turn inherits \textbf{Token}.
The casting from one type to another is done in a specific fashion
and not using the defaukt .NET casting system which is not able to
do all the casts needed. During these casts the \textbf{new} operator
is used a extensively to instantiate the new optimized token, statements
etc. and then assignments are used to transfer the information inside
the old class into the new class. The root namespace is dylan.NET.Tokenizer.Parser.


\subsection{CodeGen\label{sub:CodeGen}}

More on this after the library is actually written. The root namespace
shall be dylan.NET.Tokenizer.CodeGen.


\chapter{Other Libraries}


\section{Introduction\label{sec:Introduction-1}}

Like other programming languages dylan.NET defines its own specific
libraries. The main utility library is \textbf{dnu.dll} which contains
certain functions helping the dylan.NET programmer. Since the new
compiler is written in dylan.NET itself i.e. it is self-hosting it
makes use of dnu.dll. This means it has to be built before the compiler
if rebuilding the toolset from source. All these libraries below are
written in dylan.NET demonstrating that dylan.NET can make great libraries
like C\# can!


\section{dnu - dylan.NET Utility\label{sec:dnu---dylan.NET}}

This library provides certain constants such as pi,crlf,cr,lf,e etc.
One may say, {}``But .NET already has all that stuff''. Yes it is
true. But for now dylan.NET cannot create nor access literal fields
which means the .NET ones are useless. That is why dnu defines readonly
field versions of these constants. In the future this class may get
deprecated when compile time constants may be created and used from
dylan.NET. It is also useful to know that certain functions in dnu
might be original ones and can be needed from C\# or VB.NET. The root
namespace is dylan.NET.Utils.


\section{sld - SQLite Data\label{sec:sld---SQLite}}

This library provides an easy way to use SQLite database connections.
New datatypes storable in SQL databases can be defined and used as
file-formats. The main external data formats for dylan.NET are the
SQLite database and the XML file. For XML .NET gives us the required
abstractions but for SQLite it does not so we need wrapper libraries
such as \textbf{Mono.Data.Sqlite.dll} and a library to abstract the
wrapper such as \textbf{sld.dll}.


\section{Others...\label{sec:Others...}}

Other libraries written in dylan.NET specifically for dylan.NET may
rise in the future giving a helping hand to those who wish to invest
in dylan.NET. Note that dylan.NET made libraries are generally cross-platform
through Novell Mono. If a native library is required, the dylan.NET
library is only portable to the OSes for which a nativee library exists.
For example, sld is only portable to Windows and Linux that is why
XML will be used inside the CodeGen module.


\chapter{The Language}


\section{Introduction}

Finally here it is, the dylan.NET language manual pages.
\end{document}
